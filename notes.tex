\documentclass{article}

\usepackage{alltt}
\usepackage{amsmath}
\usepackage{amssymb}
\usepackage{amsthm}
\usepackage{array}
\usepackage[T1]{fontenc}
\usepackage{geometry}
\usepackage{graphicx}
\usepackage{multicol}
\usepackage{semantic}
\usepackage{tabularx}

\theoremstyle{definition}
\newtheorem{theorem}{Theorem}
\newtheorem{corollary}{Corollary}
\newtheorem{lemma}{Lemma}
\newtheorem{definition}{Definition}

\newcommand{\at}{\ensuremath{{\scriptstyle{@}}}}
\newcommand{\pc}{\ensuremath{{\mathit{pc}}}}

\newcommand{\CoqSymbol}
{\raisebox{-.2ex}{\includegraphics[height=0.9em]{coq.pdf}}\,}
\newcommand{\Coqed}{\hfill\CoqSymbol}

\renewcommand*{\descriptionlabel}[1]{\hspace\labelsep\normalfont #1}

\begin{document}

\begin{flushright}
  \today
\end{flushright}

\begin{figure}[ht]
  \centering
  \[
  \mathcal{L} = \{ \bot, \top \}
  \qquad
  \bot \sqsubseteq \top
  \qquad
  L^{\mathit{def}} = \bot
  \qquad
  \mathcal{A} = \mathbf{1}
  \]
  \[
  \begin{array}[t]{lcl}
    b & ::= &
    \mathit{tt}\;\; |\;\;
    \mathit{ff}
    \\[0.3ex]
    c & :: = &
    b\;\; |\;\;
    n
    \\[0.3ex]
    e & ::= &
    c\;\; |\;\;
    x\;\; |\;\;
    \lambda{x}.\, e\;\; |\;\;
    e\ e\;\; |\;\;
    \mathsf{declassify}\ e\ e
    \\[0.3ex]
    v & ::= &
    c\;\; |\;\;
    \langle{\rho, \lambda{x}.\, e\rangle}
    \\[0.3ex]
    a & ::= &
    v \at L
    \\[0.3ex]
    \rho & ::= &
    \bullet\;\; |\;\;
    \rho, x = a
  \end{array}
  \]
  \caption{Syntax of $\lambda_{\mathbf{2}}$.}
  \label{fig:syntax}
\end{figure}

\begin{definition}
  The $\approx^{L}_{P}$ relation expresses the \emph{indistinguishability} of
  atoms for observers at label $L$ and with respect to some reflexive relation
  $P$, where $v_1 \at L_1 \approx^{L}_{P} v_2 \at L_2$ iff either
  \begin{itemize}
  \item
    $L = \bot$ and
    $L_1 = L_2 = \top$ and either
    \begin{itemize}
    \item
      $v_1 = n_1$ and
      $v_2 = n_2$ and
      $P(v_1, v_2)$, or
    \item
      $v_1 = \langle{\rho_1, \lambda{x_1}.\, e_1\rangle}$ and
      $v_2 = \langle{\rho_2, \lambda{x_2}.\, e_2\rangle}$ and
      $v_1 \approx^{L}_{P} v_2$, or
    \item
      $v_1 = c_1$ and $v_2 = \langle{\rho_2,\lambda{x_2}.\, e_2\rangle}$, or
    \item
      $v_1 = \langle{\rho_1, \lambda{x_1}.\, e_1\rangle}$ and $v_2 = c_2$; or
    \end{itemize}
  \item $L = \bot$ and
    $L_1 = L_2 = \bot$ and $v_1 \approx^{L}_{P} v_2$; or
  \item $L = \top$ and $L_1 = L_2$ and $v_1 \approx^{L}_{P} v_2$.
  \end{itemize}
  The definition for atoms is mutually recursive with that for values and
  environments:
  \begin{itemize}
  \item $c_1 \approx^{L}_{P} c_2$ iff $c_1 = c_2$.
  \item
    $\langle{\rho_1, \lambda{x_1}.\, e_1\rangle} \approx^{L}_{P}
    \langle{\rho_2, \lambda{x_2}.\, e_2\rangle}$ iff
    $\rho_1 \approx^{L}_{P} \rho_2$ and
    $\lambda{x_1}.\, e_1 \equiv \lambda{x_2}.\, e_2$.
  \item $\rho_1 \approx^{L}_{P} \rho_2$ iff
    $\mathrm{dom}\; \rho_1 = \mathrm{dom}\; \rho_2$ and
    $\rho_1(x) \approx^{L}_{P} \rho_2(x)$ for every
    $x \in \mathrm{dom}\; \rho_1$.
  \end{itemize}
  We write $\approx^{L}$ when $P$ is the everywhere-true relation.
  \label{def:lp-equiv}
\end{definition}

\begin{lemma}
  $\approx^{L}_{P}$ implies $\approx^{L}$.
  \begin{enumerate}
  \item If $a_1 \approx^{L}_{P} a_2$, then $a_1 \approx^{L} a_2$.
  \item If $v_1 \approx^{L}_{P} v_2$, then $v_1 \approx^{L} v_2$.
  \item If $\rho_1 \approx^{L}_{P} \rho_2$, then $\rho_1 \approx^{L} \rho_2$.
  \Coqed
  \end{enumerate}
  \label{lem:lp-equiv-l-equiv}
\end{lemma}

\begin{lemma}
  $\approx^{L}_{P}$ is reflexive.
  \Coqed
  \label{lem:lp-equiv-refl}
\end{lemma}

\begin{lemma}
 If $v_1 \at L_1 \approx^{L}_{P} v_2 \at L_2$, then $L_1 = L_2$.
 \Coqed
 \label{lem:lp-equiv-lab-inv}
\end{lemma}

\begin{lemma}
  If $v_1 \at \bot \approx^{L}_{P} v_2 \at \bot$, then
  $v_1 \at \top \approx^{L}_{P} v_2 \at \top$.
  \Coqed
  \label{lem:lp-equiv-raise}
\end{lemma}

\pagebreak

\begin{figure}[ht]
  \centering
  \begin{gather*}
    \inference{}{
      \pc, \rho |- c \Downarrow^{k}_{l} c \at \pc
    }[\textsc{E-Const}]
    \qquad
    \inference{
      {\begin{array}{l}
          \rho(x) = v \at L
        \end{array}}
    }{
      \pc, \rho |- x \Downarrow^{k}_{l} v \at (\pc \sqcup L)
    }[\textsc{E-Var}]
    \\[3ex]
    \inference{}{
      \pc, \rho |- (\lambda{x}.\, e) \Downarrow^{k}_{l}
      \langle{\rho, \lambda{x}.\, e\rangle} \at \pc
    }[\textsc{E-Abs}]
    \qquad
    \inference{
      {\begin{array}{l}
          \pc, \rho |- e_1 \Downarrow^{k}_{l}
          \langle{\rho_1, \lambda{x}.\, e\rangle} \at L_1
          \\
          \pc, \rho |- e_2 \Downarrow^{k}_{l} a_2
          \\
          L_1, (\rho_1, x = a_2) |- e \Downarrow^{k}_{l} a_3
        \end{array}}
    }{
      \pc, \rho |- (e_1\ e_2) \Downarrow^{k+1}_{l} a_3
    }[\textsc{E-App}]
    \\[3ex]
    \inference{
      {\begin{array}{l}
          \pc, \rho |- (e_1\ e_2) \Downarrow^{k}_{l} v \at \bot
        \end{array}}
    }{
      \pc, \rho |- (\mathsf{declassify}\ e_1\ e_2) \Downarrow^{k+1}_{l}
      v \at \bot
    }[\textsc{E-Decl1}]
    \\[3ex]
    \inference{
      {\begin{array}{l}
          \pc, \rho |- e_1 \Downarrow^{k}_{l}
          \langle{\rho_1, \lambda{x}.\, e\rangle} \at L_1
          \\
          \pc, \rho |- e_2 \Downarrow^{k}_{l} a_2
          \\
          L_1, (\rho_1, x = a_2) |- e \Downarrow^{k}_{l} v_3 \at \top
          \\
          \forall{\rho_1', a_2' \text{ s.t. }
            (\rho_1, x = a_2) \approx^{L}_{P} (\rho_1', x = a_2')}.
          \\\quad
          L_1, (\rho_1', x = a_2') |- e \Downarrow^{l}_{k} v_3' \at \top =>
          v_3 \approx^{L}_{P} v_3'
        \end{array}}
    }{
      \pc, \rho |- (\mathsf{declassify}\ e_1\ e_2) \Downarrow^{k+1}_{l}
      v_3 \at \bot
    }[\textsc{E-Decl2}]
    \\[3ex]
    \pc, \rho |- e \Downarrow a
    \quad \triangleq \quad
    \exists{k}.\, \forall{l}.\,
    \pc, \rho |- e \Downarrow^{k}_{l} a
  \end{gather*}
  \caption{Semantics of $\lambda_{\mathbf{2}}$.}
  \label{fig:semantics}
\end{figure}

\begin{lemma}
  $\Downarrow^{k}_{l}\ \sqsubseteq\ \Downarrow^{k+1}_{l}$
  and
  $\Downarrow^{k}_{l+1}\ \sqsubseteq\ \Downarrow^{k}_{l}$.
  \label{lem:semantics-step}
\end{lemma}
\begin{proof}
  It suffices to show that for all $k$ and $l$ such that $k + l < n$, for
  any $n$:
  \begin{flalign}
    & \quad
    \text{If $\pc, \rho |- e \Downarrow^{k}_{l} a$, then
      $\pc, \rho |- \Downarrow^{k+1}_{l} a$.}
    \label{ih-k-k+1} & \\
    & \quad
    \text{If $\pc, \rho |- e \Downarrow^{k}_{l+1} a$, then
      $\pc, \rho |- e \Downarrow^{k}_{l} a$.}
    \label{ih-l+1-l} &
  \end{flalign}
  The proof is by induction on $n$. Note that the claim is vacuously true
  when $n = 0$. Assume the claim holds for all $k$ and $l$ such that
  $k + l < n$ for some $n$. We must show that it is true for all
  $k + l < n + 1$, that is, when $k + l = n$. We proceed by case analysis
  on the structure of $e$.

  The cases for constants $c$, variables $x$, and abstractions
  $(\lambda{x}.\, e)$ are immediate, while the case for applications
  $(e_1\ e_2)$ follows from a simple application of the induction hypothesis.
  We consider the final case for declassification operations
  $(\mathsf{declassify}\ e_1\ e_2)$ below.
  \begin{enumerate}
  \item By inversion, $k > 0$ and either:
    \begin{itemize}
    \item The derivation is by \textsc{E-Decl1}, in which case
      $\pc, \rho |- (\mathsf{declassify}\ e_1\ e_2)
      \Downarrow^{k}_{l}
      v \at \bot$ and
      $\pc, \rho |- (e_1\ e_2) \Downarrow^{k-1}_{l} v \at \bot$.
      Since $k-1 + l < n$, we can apply the induction
      hypothesis~(\ref{ih-k-k+1})
      to get
      $\pc, \rho |- (e_1\ e_2) \Downarrow^{k}_{l} v \at \bot$.
      By \textsc{E-Decl1}, we have
      \[\pc, \rho |- (\mathsf{declassify}\ e_1\ e_2) \Downarrow^{k+1}_{l}
      v \at \bot.\]
    \item The derivation is by \textsc{E-Decl2}, in which case
      $\pc, \rho |- (\mathsf{declassify}\ e_1\ e_2)
      \Downarrow^{k}_{l}
      v_3 \at \bot$ and:
      \begin{center}
        \begin{tabular*}{1.0\linewidth}{ll}
          a.
          $\pc, \rho |- e_1
          \Downarrow^{k-1}_{l}
          \langle{\rho_1, \lambda{x}.\, e\rangle} \at L_1$
          &
          b.
          $\pc, \rho |- e_2
          \Downarrow^{k-1}_{l}
          a_2$
          \\[1.8ex]
          c.
          $L_1, (\rho_1, x = a_2) |- e
          \Downarrow^{k-1}_{l}
          v_3 \at \top$
          &
          d.
          $\forall{\rho_1', a_2' \text{ s.t. }
            (\rho_1, x = a_2) \approx^{L}_{P} (\rho_1', x = a_2')}.$
          \\
          &
          \quad\quad
          $L_1, (\rho_1', x = a_2') |- e \Downarrow^{l}_{k-1} v_3' \at \top =>
          v_3 \approx^{L}_{P} v_3'$
        \end{tabular*}
      \end{center}
      Since $k-1 + l < n$, we can apply the induction
      hypothesis~(\ref{ih-k-k+1}) to a, b, and c, getting a', b', and c';
      and since $\Downarrow^{l}_{k}\ \sqsubseteq\ \Downarrow^{l}_{k-1}$ by
      the induction hypothesis~(\ref{ih-l+1-l}), we also get d':
      \begin{center}
        \begin{tabular*}{1.0\linewidth}{l@{\qquad}l}
          a'.
          $\pc, \rho |- e_1
          \Downarrow^{k}_{l}
          \langle{\rho_1, \lambda{x}.\, e\rangle} \at L_1$
          &
          b'.
          $\pc, \rho |- e_2
          \Downarrow^{k}_{l}
          a_2$
          \\[1.8ex]
          c'.
          $L_1, (\rho_1, x = a_2) |- e
          \Downarrow^{k}_{l}
          v_3 \at \top$
          &
          d'.
          $\forall{\rho_1', a_2' \text{ s.t. }
            (\rho_1, x = a_2) \approx^{L}_{P} (\rho_1', x = a_2')}.$
          \\
          &
          \quad\quad
          $L_1, (\rho_1', x = a_2') |- e \Downarrow^{l}_{k} v_3' \at \top =>
          v_3 \approx^{L}_{P} v_3'$
        \end{tabular*}
      \end{center}
      By \textsc{E-Decl2}, we have
      $\pc, \rho |- (\mathsf{declassify}\ e_1\ e_2)
      \Downarrow^{k+1}_{l}
      v_3 \at \bot$.
    \end{itemize}
  \item By inversion, $k > 0$ and either:
    \begin{itemize}
    \item The derivation is by \textsc{E-Decl1},
      in which case
      $\pc, \rho |- (\mathsf{declassify}\ e_1\ e_2)
      \Downarrow^{k}_{l+1}
      v \at \bot$ and
      $\pc, \rho |- (e_1\ e_2) \Downarrow^{k-1}_{l+1} v \at \bot$.
      Since $k-1 + l < n$, we can apply the induction
      hypothesis~(\ref{ih-l+1-l}) to get
      $\pc, \rho |- (e_1\ e_2) \Downarrow^{k-1}_{l} v \at \bot$.
      By \textsc{E-Decl1}, we have
      \[ \pc, \rho |- (e_1\ e_2) \Downarrow^{k}_{l} v \at \bot. \]
    \item The derivation is by \textsc{E-Decl2},
      in which case
      $\pc, \rho |- (\mathsf{declassify}\ e_1\ e_2)
      \Downarrow^{k}_{l+1}
      v_3 \at \bot$ and:
      \begin{center}
        \begin{tabular*}{1.0\linewidth}{ll}
          a.
          $\pc, \rho |- e_1
          \Downarrow^{k-1}_{l+1}
          \langle{\rho_1, \lambda{x}.\, e\rangle} \at L_1$
          &
          b.
          $\pc, \rho |- e_2
          \Downarrow^{k-1}_{l+1}
          a_2$
          \\[1.8ex]
          c.
          $L_1, (\rho_1, x = a_2) |- e
          \Downarrow^{k-1}_{l+1}
          v_3 \at \top$
          &
          d.
          $\forall{\rho_1', a_2' \text{ s.t. }
            (\rho_1, x = a_2) \approx^{L}_{P} (\rho_1', x = a_2')}.$
          \\ & \quad\quad
          $L_1, (\rho_1', x = a_2') |- e \Downarrow^{l+1}_{k-1} v_3' \at \top =>
          v_3 \approx^{L}_{P} v_3'$
        \end{tabular*}
      \end{center}
      Since $k-1 + l < n$, we can apply the induction
      hypothesis~(\ref{ih-l+1-l}) to a, b, and c, getting a', b', and c';
      and since $\Downarrow^{l}_{k-1} \sqsubseteq \Downarrow^{l+1}_{k-1}$
      by the induction hypothesis~(\ref{ih-k-k+1}), we also get d':
      \begin{center}
        \begin{tabular*}{1.0\linewidth}{ll}
          a'.
          $\pc, \rho |- e_1
          \Downarrow^{k-1}_{l}
          \langle{\rho_1, \lambda{x}.\, e\rangle} \at L_1$
          &
          b'.
          $\pc, \rho |- e_2
          \Downarrow^{k-1}_{l}
          a_2$
          \\[1.8ex]
          c'.
          $L_1, (\rho_1, x = a_2) |- e
          \Downarrow^{k-1}_{l}
          v_3 \at \top$
          &
          d'.
          $\forall{\rho_1', a_2' \text{ s.t. }
            (\rho_1, x = a_2) \approx^{L}_{P} (\rho_1', x = a_2')}.$
          \\ & \quad\quad
          $L_1, (\rho_1', x = a_2') |- e \Downarrow^{l}_{k-1} v_3' \at \top =>
          v_3 \approx^{L}_{P} v_3'$
        \end{tabular*}
      \end{center}
      By \textsc{E-Decl2}, we have
      $\pc, \rho |- (\mathsf{declassify}\ e_1\ e_2)
      \Downarrow^{k}_{l}
      v_3 \at \bot$.
      \qedhere
    \end{itemize}
  \end{enumerate}
\end{proof}

\begin{theorem}
  If $\pc, \rho |- e \Downarrow a$ and $\pc, \rho' |- e \Downarrow a'$ and
  $\rho \approx^{L}_{P} \rho'$, then $a \approx^{L}_{P} a'$.
  \label{thm:non-interference}
\end{theorem}
\begin{proof}\renewcommand{\qedsymbol}{}
  Unfolding the definition of $\Downarrow$, we have
  $\pc, \rho |- e \Downarrow^{k}_{l} a$
  and
  $\pc, \rho' |- e \Downarrow^{k'}_{l} a'$
  for some $k$ and $k'$ and for any $l$.
  It suffices to show that, for any $n$ such that $k+k' < n$,
  \begin{flalign}
    & \quad
    \text{If $\pc, \rho |- e \Downarrow^{k}_{l} a$ and
      $\pc, \rho' |- e \Downarrow^{k'}_{l} a'$ and
      $\rho \approx^{L}_{P} a'$, then $a \approx^{L}_{P} a'$.}
    &
  \end{flalign}
  The proof is by induction on $n$. Note that the claim is vacuously true when
  $n = 0$. Assume the claim holds for some $n$ such that $k + k' < n$. We must
  show that it is true for $k + k' < n + 1$, that is, when $k + k' = n$.
  We proceed by case analysis on the structure of $e$, inverting the
  derivations of $a$ and $a'$ at the start of each case.
  \begin{description}
  \item[\emph{Case} $c$:]
    \begin{gather*}
      \inference{}{
        \pc, \rho |- c \Downarrow^{k}_{l} c \at \pc
      }
      \quad
      \inference{}{
        \pc, \rho' |- c \Downarrow^{k'}_{l} c \at \pc
      }
    \end{gather*}
    Since $\approx^{L}_{P}$ is reflexive (Lemma~\ref{lem:lp-equiv-refl}),
    we have $c \at \pc \approx^{L}_{P} c \at \pc$.
    
  \item[\emph{Case} $x$:]
    \begin{gather*}
      \inference{
        \rho(x) = v \at L_1
      }{
        \pc, \rho |- x \Downarrow^{k}_{l} v \at (\pc \sqcup L_1)
      }
      \quad
      \inference{
        \rho'(x) = v' \at L_1'
      }{
        \pc, \rho' |- x \Downarrow^{k'}_{l} v' \at (\pc \sqcup L_1')
      }
    \end{gather*}
    Since $\rho \approx^{L}_{P} \rho'$, we have $v \at L_1 \approx v' \at L_1'$.
    By Lemma~\ref{lem:lp-equiv-lab-inv}, $L_1 = L_1'$. Therefore we must show
    that $v \at (\pc \sqcup L_1) \approx^{L}_{P} v' \at (\pc \sqcup L_1)$.
    We proceed by case analysis on $\pc$ and $L_1$.
    % 
    If $pc = \bot$, or if $\pc = \top$ and $L_1 = \top$, then
    $\pc \sqcup L_1 = L_1$ and we have
    $v \at (\pc \sqcup L_1) \approx^{L}_{P} v' \at (\pc \sqcup L_1)$
    immediately.
    %
    If $pc = \top$ and $L_1 = \bot$, then $\pc \sqcup L_1 = \pc$ and,
    by Lemma~\ref{lem:lp-equiv-raise},
    $v \at (\pc \sqcup L_1) \approx^{L}_{P} v' \at (\pc \sqcup L_1)$.
    
    % \begin{description}
    % \item[\emph{Subcase} $\pc = \bot$ and $L_1 = \bot$:]\
    %   Since $\pc \sqcup L_1 = L_1$, we have
    %   $v \at (\pc \sqcup L_1) \approx^{L}_{P} v' \at (\pc \sqcup L_1)$.
    % \item[\emph{Subcase} $\pc = \bot$ and $L_1 = \top$:]\
    %   Since $\pc \sqcup L_1 = L_1$, we have
    %   $v \at (\pc \sqcup L_1) \approx^{L}_{P} v' \at (\pc \sqcup L_1)$.
    % \item[\emph{Subcase} $\pc = \top$ and $L_1 = \bot$:]\
    %   By Lemma~\ref{lem:lp-equiv-raise},
    %   $v \at \pc \approx^{L}_{P} v' \at \pc$.
    %   Since $\pc \sqcup L_1 = \pc$, we have
    %   $v \at (\pc \sqcup L_1) \approx^{L}_{P} v' \at (\pc \sqcup L_1)$.
    % \item[\emph{Subcase} $\pc = \top$ and $L_1 = \top$:]\
    %   Since $\pc \sqcup L_1 = L_1$, we have
    %   $v \at (\pc \sqcup L_1) \approx^{L}_{P} v' \at (\pc \sqcup L_1)$.
    % \end{description}
    
  \item[\emph{Case} $(\lambda{x}.\, e)$:]
    \begin{gather*}
      \inference{}{
        \pc, \rho |- (\lambda{x}.\, e) \Downarrow^{k}_{l}
        \langle{\rho, \lambda{x}.\, e\rangle} \at \pc
      }
      \quad
      \inference{}{
        \pc, \rho' |- (\lambda{x}.\, e) \Downarrow^{k'}_{l}
        \langle{\rho', \lambda{x}.\, e\rangle} \at \pc
      }
    \end{gather*}
    
  \item[\emph{Case} $(e_1\ e_2)$:]
    \begin{gather*}
      \inference{
        {\begin{array}{l}
            \pc, \rho |- e_1 \Downarrow^{k-1}_{l}
            \langle{\rho_1, \lambda{x}.\, e\rangle} \at L_1
            \\
            \pc, \rho |- e_2 \Downarrow^{k-1}_{l} a_2
            \\
            L_1, (\rho_1, x = a_2) |- e \Downarrow^{k-1}_{l} a_3
          \end{array}}
      }{
        \pc, \rho |- (e_1\ e_2) \Downarrow^{k}_{l} a_3
      }
      \quad
      \inference{
        {\begin{array}{l}
            \pc, \rho' |- e_1 \Downarrow^{k'-1}_{l}
            \langle{\rho_1', \lambda{x'}.\, e'\rangle} \at L_1'
            \\
            \pc, \rho' |- e_2 \Downarrow^{k'-1}_{l} a_2'
            \\
            L_1', (\rho_1', x' = a_2') |- e' \Downarrow^{k'-1}_{l} a_3'
          \end{array}}
      }{
        \pc, \rho' |- (e_1\ e_2) \Downarrow^{k'}_{l} a_3'
      }
    \end{gather*}
    
  \item[\emph{Case} $(\mathsf{declassify}\ e_1\ e_2)$:]\
    
    \emph{Subcase}:
    \begin{gather*}
      \inference{
        \pc, \rho |- (e_1\ e_2) \Downarrow^{k-1}_{l} v \at \bot
      }{
        \pc, \rho |- (\mathsf{declassify}\ e_1\ e_2) \Downarrow^{k-1}_{l}
        v \at \bot
      }
      \quad
      \inference{
        \pc, \rho' |- (e_1\ e_2) \Downarrow^{k'-1}_{l} v' \at \bot
      }{
        \pc, \rho' |- (\mathsf{declassify}\ e_1\ e_2) \Downarrow^{k'-1}_{l}
        v' \at \bot
      }
    \end{gather*}

    \emph{Subcase}:
    \begin{gather*}
      \inference{
        {\begin{array}{l}
            \pc, \rho |- e_1 \Downarrow^{k-1}_{l}
            \langle{\rho_1, \lambda{x}.\, e\rangle} \at L_1
            \\
            \pc, \rho |- e_2 \Downarrow^{k-1}_{l} a_2
            \\
            L_1, (\rho_1, x = a_2) |- e \Downarrow^{k-1}_{l} v_3 \at \top
            \\
            \forall{\rho_1', a_2' \text{ s.t. }
              (\rho_1, x = a_2) \approx^{L}_{P} (\rho_1', x = a_2')}.
            \\\quad
            L_1, (\rho_1', x = a_2') |- e \Downarrow^{l}_{k-1} v_3' \at \top =>
            v_3 \approx^{L}_{P} v_3'
          \end{array}}
      }{
        \pc, \rho |- (e_1\ e_2) \Downarrow^{k}_{l} v_3 \at \bot
      }
      \\[2ex]
      \inference{
        {\begin{array}{l}
            \pc, \rho' |- e_1 \Downarrow^{k'-1}_{l}
            \langle{\rho_1', \lambda{x'}.\, e'\rangle} \at L_1'
            \\
            \pc, \rho' |- e_2 \Downarrow^{k'-1}_{l} a_2'
            \\
            L_1', (\rho_1', x' = a_2') |- e' \Downarrow^{k'-1}_{l} v_3' \at \top
            \\
            \forall{\rho_1'', a_2'' \text{ s.t. }
              (\rho_1', x' = a_2') \approx^{L}_{P} (\rho_1'', x' = a_2'')}.
            \\\quad
            L_1', (\rho_1'', x' = a_2'') |- e' \Downarrow^{l}_{k'-1}
            v_3'' \at \top =>
            v_3' \approx^{L}_{P} v_3''
          \end{array}}
      }{
        \pc, \rho' |- (e_1\ e_2) \Downarrow^{k'}_{l} v_3' \at \bot
      }
    \end{gather*}
    
  \end{description}
\end{proof}

\end{document}
