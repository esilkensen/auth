\documentclass{article}

\usepackage[includeheadfoot,margin=1in,foot=1.25in]{geometry}
\usepackage{fancyhdr}
\usepackage[T1]{fontenc}
\usepackage{units}
\renewcommand{\headrulewidth}{0pt}
\setlength{\headheight}{44pt}
\setlength{\headsep}{12pt}
\setlength{\footskip}{0.75in}

\usepackage{alltt}
\usepackage{amsmath}
\usepackage{amssymb}
\usepackage{amsthm}
\usepackage{array}
\usepackage{graphicx}
\usepackage{semantic}

\renewcommand*\descriptionlabel[1]{\hspace\labelsep\normalfont #1}

\newcommand{\at}{\ensuremath{{\scriptstyle{@}}}}
\newcommand{\pc}{\ensuremath{{\mathit{pc}}}}

\newcommand{\twoheadlongmapsto}{\mapstochar\relbar\joinrel\twoheadrightarrow}

\renewcommand*\descriptionlabel[1]{\hspace\labelsep\normalfont #1}

\newcommand{\CoqSymbol}{\raisebox{-.2ex}{\includegraphics[height=0.9em]{coq.pdf}}\,}
\newcommand{\Coqed}{\hfill\CoqSymbol}

\theoremstyle{definition}
\newtheorem{theorem}{Theorem}
\newtheorem{corollary}{Corollary}
\newtheorem{lemma}{Lemma}
\newtheorem{definition}{Definition}

\lhead{Notes: ``A Semantic Notion of Authority''}
\rhead{\today}

\begin{document}

\thispagestyle{fancy}

\begin{figure}[ht]
  \centering
  \[
  \mathcal{L} = \{ \bot, \top \}
  \qquad
  \bot \sqsubseteq \top
  \qquad
  L^{\mathit{def}} = \bot
  \qquad
  \mathcal{A} = \mathbf{1}
  \]
  \caption{Label algebra $\mathbf{2}$.}
  \label{fig:two}
\end{figure}

\begin{figure}[ht]
  \centering
  \[
  \begin{array}[t]{lcl}
    e & ::= &
    n\;\; |\;\;
    x\;\; |\;\;
    \lambda{x}.\, e\;\; |\;\;
    e\ e\;\; |\;\;
    \mathsf{declassify}\ e\ e
    \\[0.3ex]
    v & ::= &
    n\;\; |\;\;
    \langle{\rho, \lambda{x}.\, e\rangle}
    \\[0.3ex]
    a & ::= &
    v \at L
    \\[0.3ex]
    \rho & ::= &
    \bullet\;\; |\;\;
    \rho, x = a
  \end{array}
  \]
  \caption{Syntax of $\lambda_{\mathbf{2}}$.}
  \label{fig:syntax}
\end{figure}

\begin{definition}
  The $\approx^{L}_{P}$ relation expresses the \emph{indistinguishability} of
  atoms for observers at label $L$ and with respect to some reflexive relation
  $P$, where $v_1 \at L_1 \approx^{L}_{P} v_2 \at L_2$ iff either
  \begin{itemize}
  \item $L = \bot$ and
    $L_1 = L_2 = \top$ and either
    \begin{itemize}
    \item $v_1 = n_1$ and
      $v_2 = n_2$ and
      $P(n_1, n_2)$, or
    \item $v_1 = \langle{\rho_1, \lambda{x_1}.\, e_1\rangle}$ and
      $v_2 = \langle{\rho_2, \lambda{x_2}.\, e_2\rangle}$ and
      $v_1 \approx^{L}_{P} v_2$, or
    \item $v_1 = n_1$ and $v_2 = \langle{\rho_2,\lambda{x_2}.\, e_2\rangle}$, or
    \item $v_1 = \langle{\rho_1, \lambda{x_1}.\, e_1\rangle}$ and $v_2 = n_2$; or
    \end{itemize}
  \item $L = \bot$ and
    $L_1 = L_2 = \bot$ and $v_1 \approx^{L}_{P} v_2$; or
  \item $L = \top$ and $L_1 = L_2$ and $v_1 \approx^{L}_{P} v_2$.
  \end{itemize}
  The definition for atoms is mutually recursive with that for values and
  environments:
  \begin{itemize}
  \item $n_1 \approx^{L}_{P} n_2$ iff $n_1 = n_2$.
  \item
    $\langle{\rho_1, \lambda{x_1}.\, e_1\rangle} \approx^{L}_{P}
    \langle{\rho_2, \lambda{x_2}.\, e_2\rangle}$ iff
    $\rho_1 \approx^{L}_{P} \rho_2$ and
    $\lambda{x_1}.\, e_1 \equiv \lambda{x_2}.\, e_2$.
  \item $\rho_1 \approx^{L}_{P} \rho_2$ iff
    $\mathrm{dom}\; \rho_1 = \mathrm{dom}\; \rho_2$ and
    $\rho_1(x) \approx^{L}_{P} \rho_2(x)$ for every
    $x \in \mathrm{dom}\; \rho_1$.
  \end{itemize}
  We write $\approx^{L}$ when $P$ is the everywhere-true relation.
  \label{def:lp-equiv}
\end{definition}

\begin{lemma}
 If $v_1 \at L_1 \approx^{L}_{P} v_2 \at L_2$, then $L_1 = L_2$.
 \Coqed
 \label{lem:lp-equiv-lab-inv}
\end{lemma}

\begin{lemma}
  If $v_1 \at \bot \approx^{L}_{P} v_2 \at \bot$, then
  $v_1 \at \top \approx^{L}_{P} v_2 \at \top$.
  \Coqed
  \label{lem:lp-equiv-raise}
\end{lemma}

% \begin{lemma}
%   If $v_1 \at \top \approx^{\top}_{P} v_2 \at \top$, then
%   $v_1 \at \bot \approx^{\top}_{P} v_2 \at \bot$.
%   \Coqed
%   \label{lem:lp-equiv-lower}
% \end{lemma}

\begin{lemma}
  $\approx^{L}_{P}$ implies $\approx^{L}$.
  \Coqed
  % \begin{enumerate}
  % \item If $a_1 \approx^{L}_{P} a_2$, then $a_1 \approx^{L} a_2$.
  % \item If $v_1 \approx^{L}_{P} v_2$, then $v_1 \approx^{L} v_2$.
  % \item If $\rho_1 \approx^{L}_{P} \rho_2$, then $\rho_1 \approx^{L} \rho_2$.
  % \Coqed
  % \end{enumerate}
  \label{lem:lp-equiv-l-equiv}
\end{lemma}

% \begin{lemma}
%   $\approx^{\top}$ implies $\approx^{\top}_{P}$.
%   \Coqed
%   % \begin{enumerate}
%   % \item If $a_1 \approx^{\top} a_2$, then $a_1 \approx^{\top}_{P} a_2$.
%   % \item If $v_1 \approx^{\top} v_2$, then $v_1 \approx^{\top}_{P} v_2$.
%   % \item If $\rho_1 \approx^{\top} \rho_2$, then
%   %   $\rho_1 \approx^{\top}_{P} \rho_2$.
%   %   \Coqed
%   % \end{enumerate}
%   \label{lem:l-equiv-lp-equiv}
% \end{lemma}

\begin{figure}[ht]
  \centering
  \begin{gather*}
    \inference{}{
      \pc, \rho |- n \Downarrow^{k} n \at \pc
    }[\textsc{E-Nat}]
    \qquad
    \inference{
      {\begin{array}{l}
          \rho(x) = v \at L
        \end{array}}
    }{
      \pc, \rho |- x \Downarrow^{k} v \at (\pc \sqcup L)
    }[\textsc{E-Var}]
    \\[3ex]
    \inference{}{
      \pc, \rho |- (\lambda{x}.\, e) \Downarrow^{k}
      \langle{\rho, \lambda{x}.\, e\rangle} \at \pc
    }[\textsc{E-Abs}]
    \qquad
    \inference{
      {\begin{array}{l}
          \pc, \rho |- e_1 \Downarrow^{k}
          \langle{\rho_1, \lambda{x}.\, e\rangle} \at L_1
          \\
          \pc, \rho |- e_2 \Downarrow^{k} a_2
          \\
          L_1, (\rho_1, x = a_2) |- e \Downarrow^{k} a_3
        \end{array}}
    }{
      \pc, \rho |- (e_1\ e_2) \Downarrow^{k+1} a_3
    }[\textsc{E-App}]
    \\[3ex]
    \inference{
      {\begin{array}{l}
          \pc, \rho |- (e_1\ e_2) \Downarrow^{k} v \at \bot
        \end{array}}
    }{
      \pc, \rho |- (\mathsf{declassify}\ e_1\ e_2) \Downarrow^{k+1} v \at \bot
    }[\textsc{E-Decl1}]
    \\[3ex]
    \inference{
      {\begin{array}[b]{l}
          \pc, \rho |- e_1 \Downarrow^{k'}
          \langle{\rho_1, \lambda{x}.\, e\rangle} \at L_1
          \\
          \pc, \rho |- e_2 \Downarrow^{k'} a_2
          \\
          L_1, (\rho_1, x = a_2) |- e \Downarrow^{k'} v_3 \at \top
          \\
          \forall{\rho_1', a_2' \text{ s.t. }
            (\rho_1, x = a_2) \approx^{L}_{P} (\rho_1', x = a_2')}.\,
          \\\quad
          L_1, (\rho_1', x = a_2') |- e \Downarrow^{k'} v_3' \at \top =>
          v_3 \approx^{L}_{P} v_3'
          \\
          k' \leq k
        \end{array}}
    }{
      \pc, \rho |- (\mathsf{declassify}\ e_1\ e_2) \Downarrow^{k+1} v_3 \at \bot
    }[\textsc{E-Decl2}]
  \end{gather*}
  \caption{Semantics of $\lambda_{\mathbf{2}}$.}
  \label{fig:semantics}
\end{figure}

% \pagebreak

\begin{lemma}
  If $\pc, \rho |- e \Downarrow^{k} a$, then
  $\pc, \rho |- e \Downarrow^{k+1} a$.
  \label{lem:semantics-ind}
\end{lemma}
\begin{proof}
  By induction on $k$.
  %The base case is trivial: if $k=0$, then a derivation
  %of $\pc, \rho |- e \Downarrow^{k} a$ must be from either
  %\textsc{E-Nat}, \textsc{E-Var}, or \textsc{E-Abs}, and
  %$\pc, \rho |- e \Downarrow^{k+1} a$ follows by definition for all three.
  % 
  % For the inductive case, suppose $\pc, \rho |- e \Downarrow^{k} a$ implies
  % $\pc, \rho |- e \Downarrow^{k+1} a$ for some $k$. By case analysis on a
  % derivation of $\pc, \rho |- e \Downarrow^{k} a$, we have
  % $\pc, \rho |- e \Downarrow^{k+2} a$ by definition for
  % \textsc{E-Nat}, \textsc{E-Var}, \textsc{E-Abs}, and \textsc{E-Decl2},
  % and by the induction hypothesis for \textsc{E-App} and \textsc{E-Decl1}.
\end{proof}

\begin{lemma}
  If $\pc, \rho |- e \Downarrow^{k} a$ and
  $\pc, \rho' |- e \Downarrow^{k} a'$
  and $\rho \approx^{L}_{P} \rho'$, then
  $a \approx^{L}_{P} a'$.
\end{lemma}
\begin{proof}
  By induction on a derivation of $\pc, \rho |- e \Downarrow^{k} a$.
  \begin{description}
  \item[\emph{Case} \textsc{E-Nat}:]
    \begin{gather*}
      \inference{}{
        \pc, \rho |- n \Downarrow^{k} n \at \pc
      }
      \quad
      \inference{}{
        \pc, \rho' |- n \Downarrow^{k} n \at \pc
      }
    \end{gather*}
    By case analysis on $L$ and $\pc$.
  \item[\emph{Case} \textsc{E-Var}:]
    \begin{gather*}
      \inference{
        {\begin{array}{l}
            \rho(x) = v \at L_1
          \end{array}}
      }{
        \pc, \rho |- x \Downarrow^{k} v \at (\pc \sqcup L_1)
      }
      \quad
      \inference{
        {\begin{array}{l}
            \rho'(x) = v' \at L_1'
          \end{array}}
      }{
        \pc, \rho' |- x \Downarrow^{k} v' \at (\pc \sqcup L_1')
      }
    \end{gather*}
    By case analysis on $\pc$ and $L_1$,
    using Lemmas~\ref{lem:lp-equiv-lab-inv}~and~\ref{lem:lp-equiv-raise}.
  \item[\emph{Case} \textsc{E-Abs}:]
    \begin{gather*}
      \inference{}{
        \pc, \rho |- (\lambda{x}.\, e) \Downarrow^{k}
        \langle{\rho, \lambda{x}.\, e\rangle} \at \pc
      }
      \quad
      \inference{}{
        \pc, \rho' |- (\lambda{x}.\, e) \Downarrow^{k}
        \langle{\rho', \lambda{x}.\, e\rangle} \at \pc
      }
    \end{gather*}
    By case analysis on $L$ and $\pc$.
  \item[\emph{Case} \textsc{E-App}:]
    \begin{gather*}
      \inference{
        {\begin{array}{l}
            \pc, \rho |- e_1 \Downarrow^{k}
            \langle{\rho_1, \lambda{x}.\, e\rangle} \at L_1
            \\
            \pc, \rho |- e_2 \Downarrow^{k} a_2
            \\
            L_1, (\rho_1, x = a_2) |- e \Downarrow^{k} a_3
          \end{array}}
      }{
        \pc, \rho |- (e_1\ e_2) \Downarrow^{k+1} a_3
      }
      \quad
      \inference{
        {\begin{array}{l}
            \pc, \rho' |- e_1 \Downarrow^{k}
            \langle{\rho_1', \lambda{x'}.\, e'\rangle} \at L_1'
            \\
            \pc, \rho' |- e_2 \Downarrow^{k} a_2'
            \\
            L_1', (\rho_1', x' = a_2') |- e' \Downarrow^{k} a_3'
          \end{array}}
      }{
        \pc, \rho' |- (e_1\ e_2) \Downarrow^{k+1} a_3'
      }
    \end{gather*}
  \item[\emph{Case} \textsc{E-Decl1}:]
    \begin{gather*}
      \inference{
        {\begin{array}{l}
            \pc, \rho |- (e_1\ e_2) \Downarrow^{k} v \at \bot
          \end{array}}
      }{
        \pc, \rho |- (\mathsf{declassify}\ e_1\ e_2) \Downarrow^{k+1}
        v \at \bot
      }
      \quad
      \inference{
        {\begin{array}{l}
            \pc, \rho' |- (e_1\ e_2) \Downarrow^{k} v' \at \bot
          \end{array}}
      }{
        \pc, \rho' |- (\mathsf{declassify}\ e_1\ e_2) \Downarrow^{k+1}
        v' \at \bot
      }
    \end{gather*}
  \item[\emph{Case} \textsc{E-Decl2}:]
    \begin{small}
    \begin{gather*}
      \inference{
        {\begin{array}[b]{l}
            \pc, \rho |- e_1 \Downarrow^{k'}
            \langle{\rho_1, \lambda{x}.\, e\rangle} \at L_1
            \\
            \pc, \rho |- e_2 \Downarrow^{k'} a_2
            \\
            L_1, (\rho_1, x = a_2) |- e \Downarrow^{k} v_3 \at \top
            \\
            \forall{\rho_1', a_2' \text{ s.t. }
              (\rho_1, x = a_2) \approx^{L}_{P} (\rho_1', x = a_2')}.\,
            \\\quad
            L_1, (\rho_1', x = a_2') |- e \Downarrow^{k'} v_3' \at \top =>
            v_3 \approx^{L}_{P} v_3'
            \\
            k' \leq k
          \end{array}}
      }{
        \pc, \rho |- (\mathsf{declassify}\ e_1\ e_2) \Downarrow^{k+1}
        v_3 \at \bot
      }
      \quad
      \inference{
        {\begin{array}[b]{l}
            \pc, \rho' |- e_1 \Downarrow^{k'}
            \langle{\rho_1', \lambda{x'}.\, e'\rangle} \at L_1'
            \\
            \pc, \rho' |- e_2 \Downarrow^{k'} a_2'
            \\
            L_1',  (\rho_1', x' = a_2') |- e' \Downarrow^{k'} v_3' \at \top
            \\
            \forall{\rho_1'', a_2'' \text{ s.t. }
              (\rho_1', x' = a_2') \approx^{L}_{P} (\rho_1'', x' = a_2'')}.
            \\\quad
            L_1', (\rho_1'', x' = a_2'') |- e' \Downarrow^{k'} v_3'' \at \top =>
            v_3' \approx^{L}_{P} v_3''
            \\
            k' \leq k
          \end{array}}
      }{
        \pc, \rho' |- (\mathsf{declassify}\ e_1\ e_2) \Downarrow^{k+1}
        v_3' \at \bot
      }
    \end{gather*}
    \end{small}
    % 
    % By the induction hypothesis,
    % $\langle{\rho_1, \lambda{x}.\, e\rangle} \at L_1
    % \approx^{L}_{P}
    % \langle{\rho_1', \lambda{x'}.\, e'\rangle} \at L_1'$
    % and
    % $a_2 \approx^{L}_{P} a_2'$.
    % By inversion of the definition of $\approx^{L}_{P}$,
    % $L_1 = L_1'$ and
    % $\lambda{x}.\, e = \lambda{x'}.\, e'$ and
    % $\rho_1 \approx^{L}_{P} \rho_1'$.
    % By the definition of $\approx^{L}_{P}$,
    % $(\rho_1, x = a_2) \approx^{L}_{P} (\rho_1', x' = a_2')$.
    % Since $L_1, (\rho_1', x = a_2') |- e \Downarrow^{k'} v_3' \at \top$,
    % it follows that $v_3 \approx^{L}_{P} v_3'$ and therefore
    % $v_3 \at \bot \approx^{L}_{P} v_3' \at \bot$.
  \end{description}
\end{proof}

\end{document}
